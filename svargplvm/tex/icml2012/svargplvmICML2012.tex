
%%%%%%%%%%%%%%%%%%%%%%%%%%%%%%%%%%%%%%%%%%%%%%%%%%%%%%%%%%%%%%%%%%
%%%%%%%% ICML 2012 EXAMPLE LATEX SUBMISSION FILE %%%%%%%%%%%%%%%%%
%%%%%%%%%%%%%%%%%%%%%%%%%%%%%%%%%%%%%%%%%%%%%%%%%%%%%%%%%%%%%%%%%%

% Use the following line _only_ if you're still using LaTeX 2.09.
%\documentstyle[icml2012,epsf,natbib]{article}
% If you rely on Latex2e packages, like most moden people use this:
\documentclass{article}

% For figures
\usepackage{graphicx} % more modern
%\usepackage{epsfig} % less modern
\usepackage{subfigure} 

% For citations
\usepackage{natbib}

% For algorithms
\usepackage{algorithm}
\usepackage{algorithmic}

% As of 2011, we use the hyperref package to produce hyperlinks in the
% resulting PDF.  If this breaks your system, please commend out the
% following usepackage line and replace \usepackage{icml2012} with
% \usepackage[nohyperref]{icml2012} above.
\usepackage{hyperref}

% Packages hyperref and algorithmic misbehave sometimes.  We can fix
% this with the following command.
\newcommand{\theHalgorithm}{\arabic{algorithm}}

% Employ the following version of the ``usepackage'' statement for
% submitting the draft version of the paper for review.  This will set
% the note in the first column to ``Under review.  Do not distribute.''
%\usepackage{icml2012} 
% Employ this version of the ``usepackage'' statement after the paper has
% been accepted, when creating the final version.  This will set the
% note in the first column to ``Appearing in''
\usepackage[accepted]{icml2012}

%-------
\usepackage{times}
\usepackage{epsfig}
\usepackage{graphicx}
\usepackage{amsmath}
\usepackage{amssymb}
\usepackage{subfigure}
\usepackage{cancel}
\usepackage{color}
\usepackage{hyperref}
%\usepackage{multicol}
%\usepackage{blindtext}

%\usepackage[usenames,dvipsnames]{color}

\newcommand{\highlight}[1]{\colorbox{yellow}{#1}}
\newcommand{\bff}{\mathbf{f}}
\newcommand{\bfu}{\mathbf{u}}
\newcommand{\bfy}{\mathbf{y}}
\newcommand{\bfx}{\mathbf{x}}
\newcommand{\bft}{\mathbf{t}}
\newcommand{\bfk}{\mathbf{k}}
\newcommand{\bfzi}{\mathbf{z}}
\newcommand{\bfw}{\mathbf{w}}
\newcommand{\bfmu}{\boldsymbol \mu}
\newcommand{\bfz}{\mathbf{0}}
\newcommand{\bftheta}{\boldsymbol \theta}
\newcommand{\T}{{\top}}
\newcommand{\bfa}{\mathbf{a}}
\newcommand{\bb}{\beta^{-1}}
\newcommand{\la}{\left\langle}
\newcommand{\ra}{\right\rangle}
\newcommand{\vv}{\vartheta}
\newcommand{\intd}{\text{d}}
\newcommand{\ie}{i.e.\ }
\newcommand{\eg}{e.g.\ }

%------------
\definecolor{light-gray}{gray}{0.43}



% The \icmltitle you define below is probably too long as a header.
% Therefore, a short form for the running title is supplied here:
\icmltitlerunning{Manifold Relevance Determination}


\begin{document}

\twocolumn[
\icmltitle{Manifold Relevance Determination}

% It is OKAY to include author information, even for blind
% submissions: the style file will automatically remove it for you
% unless you've provided the [accepted] option to the icml2012
% package.
%\icmlauthor{Andreas C. Damianou$^1$}{Andreas.Damianou@sheffield.ac.uk} \\
%%\icmladdress{Dept. of Computer Science \& Sheffield Institute for Translational Neuroscience, University of Sheffield, UK}
%\icmlauthor{Carl Henrik Ek}{chek@csc.kth.se}
%\icmladdress{KTH -- Royal Institute of Technology, CVAP Lab, Stockholm, Sweden}
%\icmlauthor{Michalis K. Titsias}{mtitsias@well.ox.ac.uk}
%\icmladdress{Wellcome Trust Centre for Human Genetics, Roosevelt Drive, Oxford OX3 7BN, UK }
%\icmlauthor{Neil D. Lawrence$^2$}{n.lawrence@sheffield.ac.uk}
%\icmladdress{$^{1,2}$Dept. of Computer Science \& Sheffield Institute for Translational Neuroscience, University of Sheffield, UK}

\icmlauthor{Andreas C. Damianou}{Andreas.Damianou@sheffield.ac.uk}
\icmladdress{Dept. of Computer Science \& Sheffield Institute for Translational Neuroscience, University of Sheffield, UK}
\icmlauthor{Carl Henrik Ek}{chek@csc.kth.se}
\icmladdress{KTH -- Royal Institute of Technology, CVAP Lab, Stockholm, Sweden}
\icmlauthor{Michalis K. Titsias}{mtitsias@well.ox.ac.uk}
\icmladdress{Wellcome Trust Centre for Human Genetics, Roosevelt Drive, Oxford OX3 7BN, UK }
\icmlauthor{Neil D. Lawrence}{n.lawrence@sheffield.ac.uk}
\icmladdress{Dept. of Computer Science \& Sheffield Institute for Translational Neuroscience, University of Sheffield, UK}

% You may provide any keywords that you 
% find helpful for describing your paper; these are used to populate 
% the "keywords" metadata in the PDF but will not be shown in the document
\icmlkeywords{GP-LVM, Bayesian, machine learning, ICML}

\vskip 0.3in
]

% \begin{abstract} 
%   In this paper we present a fully Bayesian latent variable model
%   which exploits conditional non-linear (in)-dependence structures to
%   learn an efficient latent representation. The model is capable of
%   learning from extremely high-dimensional data such as directly
%   modelling high resolution images. The latent representation is
%   factorized to represent shared and private information from multiple
%   views of the data. Bayesian techniques allow us to automatically
%   estimate the dimensionality of the latent spaces. We demonstrate the
%   model by prediction of human pose in an ambiguous setting. Our
%   Bayesian representation allows us to perform disambiguation in a
%   principled manner by including priors which incorporate the dynamic
%   structure of the data. 
%   We demonstrate the ability of the model to
%   capture structure underlying extremely high dimensional spaces by
%   learning a low-dimensional representation of a set of facial images
%   under different illumination conditions. The model correctly
%   automatically creates a factorized representation where the lighting
%   variance is represented in a separate latent space from the variance
%   associated with different faces. We show that the model is capable
%   of generating morphed faces and images from novel light directions.
% \end{abstract} 

\begin{abstract} 
  In this paper we present a fully Bayesian latent variable model
  which exploits conditional non-linear (in)-dependence structures to
  learn an efficient latent representation. 
  The latent space is factorized to represent shared and private information from multiple
  views of the data. In contrast to previous approaches, we introduce a relaxation to
  the discrete segmentation and allow for a ``softly'' shared latent space.
  Further, Bayesian techniques allow us to automatically
  estimate the dimensionality of the latent spaces.
%
%   The model is capable of
%    capturing structure underlying extremely high dimensional spaces.
%   This is illustrated by directly modelling the pixels of a set of high resolution facial images
%   under different illumination conditions. We show that the model automatically
%   correctly separates the lighting variance from the one associated with different
%   face characteristics and is also able of generating morphed faces and images from novel light directions.
 The model is capable of capturing structure underlying extremely high dimensional spaces.
 This is illustrated by 
% directly modelling relatively large 
 modelling unprocessed images with tenths of thousands of pixels. This also allows us
 to directly generate novel images from the trained model by sampling from the discovered latent spaces.
%
  We also demonstrate the
  model by prediction of human pose in an ambiguous setting. Our
  Bayesian framework allows us to perform disambiguation in a
  principled manner by including latent space priors which incorporate the dynamic
  nature of the data. 
\end{abstract} 




%%%%%%%%% BODY TEXT
\section{Introduction}

%---------------------------------- INTRODUCTION ------------------------------------------------------------------
\section{Introduction}
%\begin{verbatim}
%1. dimensionality reduction
%2. gplvm, applications, dynamics etc 
%3. training: so far, all of the above have been trained with MAP
%4. We present... . Advantages: prior on the latent space allows to capture assumptions and prior knowledge (e.g. dynamics),
% fully bayesian  approach resists overfitting, learn automatically the dimensionality of the latent space.
%* The remainder of this paper is structured as follows:
%
%
%1.
%---- Plan 1 
%dimensionality reduction to model a set of observations
%spectral vs probabilistic
%linear vs nonlinear
%gplvm
%gplvm extensions and applications (incl. dynamics etc, briefly).
%our contribution: a variational framework that allows for bayesian treatment of the gplvm including the dynamics extension
%
%--
%* Nonlinear dimensionality reduction:
%  Kernel PCA (non-probabilistic) etc
%* MCMC methods?
%* papers that use GPLVM with MAP
%... and applications.
%
%---- Plan 2
%Gaussian processes , regression ... ?
%
%-------------
%
%
%\end{verbatim}

%Modelling high dimensional datasets constitutes a key challenge for the machine learning community, since capturing
% the structure of the data is not easy in many dimensions.
High dimensional data are endemic in applications of machine learning.
  A typical starting point when dealing with such problems,
 is to assume that the data can be represented in a lower dimensional subspace immersed in the original, high dimensional
 one. Many successful methods (\eg PCA) seek a low dimensional manifold which is a linear subspace of the 
original data. However, this kind of linearity often constitutes a crude assumption for high dimensional and real world 
data. Here, we focus on non-linear dimensionality reduction methods.


Given a set of data, a large
range of non-probabilistic dimensionality reduction approaches have been
suggested, ranging from spectral methods such as KPCA \citep{Scholkopf:kernelpca97}, Isomap \citep{Tenenbaum:isomap00}
and other methods based on Multi-Dimensional Scaling \citep{Mardia:multivariate79}, to  iterative methods, such as Sammon mappings
\citep{Sammon:nonlinear69} and to other local distance preservation methods, such as LLE \citep{Roweis:lle00}.

Probabilistic approaches, such as GTM \citep{Bishop:gtm_ncomp98} and Density Networks \citep{MacKay:wondsa95}, view the dimensionality
reduction problem under a different perspective, since they seek a mapping from a low-dimensional latent space
to the observed data space, and come with certain advantages. More precisely, their generative nature and the forward mapping
that they define, allows them to be 
extended more easily in various ways (\eg with additional dynamics modelling), to be incorporated into a Bayesian 
framework for parameter learning and to handle more naturally missing data.

%often based on
%Multi-Dimensional Scaling [4], to generative methods such
%as GTM [3] and density networks. \cite{}.


%--
%\par In general, dimensionality reduction algorithms can be roughly classified as spectral or probabilistic. Spectral 
%approaches seek to represent the original $N$ datapoints by finding the eigendecomopsition of a (centered) similarity
% $N \times N$ matrix $K$ which summarizes the data structure. This is basically the main idea behind Classical 
%Multidimensional Scaling \cite{} which, under a certain viewpoint, can be seen as the root of many more recent 
%approaches \cite{} which mainly differ in the metric used to compute the similarity matrix or in the incorporation 
%of a mapping between the low and the high dimensional spaces. 
%%For example, Isomap \cite{} is differed from MDS in 
%%that the distance measure used seeks to preserve ... . Kernel PCA \cite{} .... . However, given that the eigenvectors
%% are never computed in the  ... real dimesinoality reductino is not always achieved.... . LLE .... . 
%\highlight{TODO:} Isomap, KPCA, LLE...

%\par
%% [In contrast, a probabilistic interpretation of dimensionality reduction problems results in models which have 
%%different (and for some applications more attractive) advantages.].
% Probabilistic approaches view the dimensionality
% reduction problem under a different perspective and come with certain advantages. More precisely, they can be 
%extended more easily in various ways (e.g. with additional dynamics modelling), can be incorporated into a Bayesian 
%framework for parameter learning and can handle more naturally missing data. \highlight{TODO:} GTM and Density networks ... .
%---


\par The Gaussian Process Latent Variable Model (GP-LVM) \citep{GPLVM} is a more recent probabilistic dimensionality 
reduction method which has been proven to be very robust for high dimensional problems 
\citep{gplvmLarger, Damianou:vgpds11}. GP-LVM can be seen 
as a non-linear generalisation of PPCA \citep{Tipping:probpca99}, which also has a
 Bayesian interpretation \citep{Bishop:bayesPCA98}. 
Unlike PPCA, though, the non-linear mapping of GP-LVM makes a Bayesian
treatment much more challenging.
Therefore, GP-LVM itself and all of its extensions, rely on a MAP training procedure.
 However, a principled Bayesian formulation is highly desirable, since it would allow for robust training of the model,
 automatic selection of the latent space's dimensionality as well as more intuitive exploration of the latent space's 
structure.

\par In this paper we formulate a variational inference framework which allows us to integrate out the inputs of the
Gaussian process and compute a lower bound on the exact marginal likelihood of the nonlinear latent variable model. The
procedure followed here is non-standard, as computation of a closed-form Jensen's lower bound on the true log marginal
 likelihood of the data is infeasible with standard variational inference. Instead, 
we build on, and significantly extend, the variational sparse GP method of \cite{Titsias:variational09}, where the GP prior is augmented to include auxiliary inducing variables so that the approximation is applied on an expanded probability model. 
\todo{expand a bit, talk also about the uncertain inputs and the flexibility of the model}


\par In the remainder of this paper we review the basics of GP-LVMs and its dynamical extensions,
 in section \ref{section:background}, and we proceed in presenting our variational framework and 
 Bayesian training procedure in section \ref{section:vgplvm}. In section \ref{section:predictions} we
 describe how the resulting model can be used for various kinds of predictive tasks and in section \ref{section:extensions}
we discuss natural but important extensions of our model. In section \ref{section:experiments}, we
 describe the experiments conducted on real world datasets and based on the results and
the analysis performed there, we present our final conclusions
in section \ref{section:conclusion}.

% \section{Related Work}



\section{The Model \label{model}}
%\subsection{Automatic discovery of a factorised latent space}

We wish to relate two views $Y \in \mathbb{R}^{N\times D_Y}$ and
$Z\in\mathbb{R}^{N\times D_Z}$ of a dataset within the same model. We
assume the existence of a single latent variable $X\in
\mathbb{R}^{N\times Q}$ which, through the mappings
$\{f^Y_d\}_{d=1}^{D_Y}: X \mapsto Y$ and $\{f^Z_d\}_{d=1}^{D_Z}: X
\mapsto Z$ ($Q<D$), gives a low dimensional representation of the
data. Our assumption is that the data is generated from a low
dimensional manifold and corrupted by additive Gaussian noise
$\epsilon^{\{Y,Z\}} \sim \mathcal{N}(\bfz, \sigma^{\{Y,Z\}}_{\epsilon} I)$,
%$\epsilon$
\begin{align}
  y_{nd} &= f^Y_d(\mathbf{x}_n) + \epsilon^Y_{nd}\nonumber\\
  z_{nd} &= f^Z_d(\mathbf{x}_n) + \epsilon^Z_{nd},
\end{align}
where $\{y,z\}_{nd}$ represents dimension $d$ of point $n$.  This
leads to the likelihood under the model, $P(Y,Z|X,\bftheta)$, where
$\bftheta = \{\bftheta^Y, \bftheta^Z\}$ collectively denotes the 
%parameters of the mappings.  
parameters of the mapping functions and the noise variances $\sigma_{\epsilon}^{\{Y,Z\}}$.
Finding the latent representation $X$ and the mappings
$f^Y$ and $f^Z$ is an ill-constrained problem. \citet{Lawrence:2005vk}
suggested regularizing the problem by placing Gaussian process (GP)
\cite{Rasmussen:book06} priors over the mappings and the resulting models
are known as Gaussian Process latent variable models (GP-LVMs).

In the GP-LVM framework each generative mapping is modeled as a
product of independent GP's parametrized by a (typically shared)
covariance function $k^{\{Y,Z\}}$ evaluated over the latent variable
$X$, so that
\begin{equation}
\label{pfx}
p(F^Y|X, \bftheta^Y) = \prod_{d=1}^{D_Y} \mathcal{N}(\bff^Y_d | \bfz, K^Y) ,
\end{equation}
where $F^Y = \{\bff^Y_d\}_{d=1}^{D_Y}$ with $f^Y_{nd} = f^Y_d(\bfx_n)$, and
similarly for $F^Z$.
%% :
%% \begin{equation}
%% \label{eq:GPpriors}
%%  f^{\{Y,Z\}}_d(\bfx)  \sim \mathcal{GP}\left(0, k^{\{Y,Z\}}(\bfx_i, \bfx_j) \right),
%%  %f^Z_d(\bfx) & \sim \mathcal{GP}\left(0, k^Z(\bfx_i, \bfx_j) \right)
%% \end{equation}
%% where the covariance functions $k^Y$ and $k^Z$ define the properties of the mappings.
This allows for general non-linear mappings to be marginalised out
analytically leading to a likelihood as a product of Gaussian
densities,
\begin{equation}
\resizebox{.9\hsize}{!}{$P(Y,Z | X,\bftheta) = 
  \prod_{\mathcal{K} = \{Y,Z\}} \int p(\mathcal{K} | F^\mathcal{K})
   p(F^\mathcal{K} | X, \bftheta^\mathcal{K}) \intd F^\mathcal{K}.$} \label{eq:likelihood}
    %& \int p(Y | F^Y) p(F^Y | X, \bftheta^Y) \intd F^Y \cdot \nonumber \\
	%&\int p(Z | F^Z) p(F^Z | X, \bftheta^Z) \intd F^Z. \label{eq:likelihood}
\end{equation}
%
%\begin{align}
%P(Y,Z|X,\bftheta) = 
%\prod_{n=1}^N \Big( 
%    & \int p(\bfy_n|\bfx_n, \bftheta^Y) \intd f^Y \cdot \nonumber \\
%	& \int p(\bfzi_n|\bfx_n, \bftheta^Z) \intd f^Z \Big). \label{eq:likelihood}
%\end{align}
%
A fully Bayesian treatment requires integration over the latent
variable $X$ in equation \eqref{eq:likelihood} which is intractable,
as $X$ appears non-linearly in the inverse of the covariance matrices
$K^Y$ and $K^Z$ of the GP priors for $f^Y$ and $f^Z$. In practice, a
maximum a posteriori solution
\cite{Shon:2006wr,Ek:2007uo,Salzmann:2010vh} was often used. However,
failure to marginalize out the latent variables means that it is not
possible to automatically estimate the dimensionality of the latent space or the
parameters of any prior distributions used in the latent space. We
show how we can obtain an approximate Bayesian training and inference
procedure by variationally marginalizing out $X$. 
%This allows us to
%estimate the latent space's dimensionality and the prior distributions over
%the latent variables. 
We achieve this by building on recent
variational approximations for standard GP-LVMs
\cite{Titsias:bayesGPLVM10,Damianou:vgpds11}. We then introduce
\emph{automatic relevance determination} (ARD) priors
\cite{Rasmussen:book06} so that each view of the data is allowed to estimate
a separate vector of ARD parameters. This allows the views to determine
which of the emerging private and shared latent spaces are relevant to
them. We refer to this idea as \emph{manifold relevance determination}
(MRD).% to be used for the mappings.  This is central
% to our model as it allows each dimension to be associated with a
% weight which defines its ``importance'' with respect to reconstructing
% the observed data. In \cite{Titsias:bayesGPLVM10} this was used as a
% means of learning the dimensionality of the latent space directly from
% data while here we additionally exploit this to learn a factorization
% of the latent variable.

\subsection{Manifold Relevance Determination}
% Force figure to be come a page earlier
%
\begin{figure*}[t]
  \begin{center}
    \includegraphics[width=0.7\textwidth]{../diagrams/graphicalmodel.pdf}
  \end{center}
 % \vspace{-8pt}
  \caption{ %\small {\it
    Evolution of the structure of GP-LVM model variants. \emph{Far left:}
    \citet{Lawrence:2005vk}'s original model is shown, a single latent
    variable $X$ is used to represent the observed data $Y$. Evolved
    shared models then (\emph{left} to \emph{right}) assume firstly,
    that all of the variance in the observations was shared \cite{Shon:2006wr}. Secondly, \citet{Ek:2008up} introduced private latent spaces to explain variance specific to one of the views. MAP estimates used in this model meant the structure of the latent space could not be automatically determined. The rightmost image shows the model we propose in this
    paper.
%
%Here, the kernel hyperparameters are actually $\bftheta^{\{Y,Z\}} = \{ \sigma_{ard}^{ \{Y,Z\} }, \bfw^{\{Y,Z\}} \}$,
%but in the figure we explicitly used $\bfw^{\{Y,Z\}}$, to emphasize the usage of ARD covariance functions.    
%
     In this figure we have separated the ARD weights $\bfw^{\{Y,Z\}}$ from the full set of model hyperparameters
      $\bftheta^{\{Y,Z\}} = \{ \sigma_{\epsilon}^{\{Y,Z\}}, \sigma_{ard}^{\{Y,Z\}}, \bfw^{\{Y,Z\}} \}$, 
      just to emphasize the usage of ARD covariance functions.
 %   
     The latent space $X$ is marginalised out and we learn a
    distribution of latent points for which additional hyperparamters
    encode the relevance of each dimension independently for the
    observation spaces and, thus, automatically define a factorisation
    of the data. The distribution $p(X) = p(X|\bftheta^X)$ placed on the latent space also enables the incorporation of prior knowledge about its structure.
      %By
      %introducing additional hyperparameters to the model encoding the
      %relevance of each dimension independently for the observation
      %spaces we can automatically learn the factorization of the data.
 % }
  }
 \label{fig:grModel}
\end{figure*}
We wish to recover a factorized latent representation such that the
variance shared between different observation spaces can be aligned
and separated from variance that is specific (private) to the separate
views.  In manifold relevance determination the notion of a hard
separation between private and shared spaces is relaxed to a
continuous setting. The model is allowed to (and indeed often does)
completely allocate a latent dimension to private or shared spaces,
but may also choose to endow a shared latent dimension with more or
less relevance for a particular data-view. Importantly, this
factorization is learned from data by maximizing a variational lower
bound on the model evidence, rather than through construction of
bespoke regularizers to achieve the same
effect.  
%
The model we propose can be seen as a generalisation of the traditional
approach to manifold learning; we still
assume the existence of a low-dimensional representation encoding the
underlying phenomenon, but the variance contained in an observation
space does not necessarily need to be governed by the full manifold,
as traditionally assumed, nor by a subspace geometrically orthogonal
to that, as assumed in \citet{Salzmann:2010vh}.
%
% What makes this possible is the
% marginalisation of the latent space $X$.  Without this the likelihood
% would always increase by the use of additional latent dimensions,
% since the number of model parameters increases proportionally to $Q$,
% and the ARD weights would be unable to ``switch off'' dimensions that
% are irrelevant for a specific observation space.

%%  to individually 
%% the relevance of each latent dimension therefore selecting a
%% ``smooth'' subspace of the manifold 

%%  allowing each generative mapping to separately determine the
%% relevance of each latent dimension.
%% %This allows us to learn a
%% %factorized latent space where a dimension considered important, as
%% %indicated by the ARD weight, by more than one observation space is
%% %considered shared and a dimension considered important by a single
%% %space as private.  
%% %
%% This means that we automatically learn a ``soft'' factorization
%% of the latent space without having to rely on hand crafted
%% regularizers such as in \cite{Salzmann:2010vh}. 
%% %
%% What makes this possible is the marginalisation of the latent space
%% $X$.  Without this the likelihood would always increase by the use of
%% additional latent dimensions, since the number of model parameters
%% increases proportionally to $Q$, and the ARD weights would be unable
%% to ``switch off'' dimensions that are irrelevant for a specific
%% observation space.


%
%%% There is some redundancy here...
%
%The traditional approach to manifold learning has been to assume
%that the observed high-dimensional data have been generated from a
%low-dimensional latent variable. The model we propose  %in this paper
%can be seen as a generalisation of this, as we still
%
%
The expressive power of our model comes from the ability to consider
non-linear mappings within a Bayesian framework.  Specifically, our
$D_Y$ latent functions $f^Y_d$ are selected to be independent draws of
a zero-mean GP with an ARD covariance function of the form:
\begin{align}
  k^Y \left( \bfx_i, \bfx_j \right) = {} & (\sigma_{ard}^Y)^2 e^{ -
    \frac{1}{2} \sum_{q=1}^{Q} w^Y_q \left( x_{i,q} - x_{j,q} \right)
    ^2 },
\label{ARDkernel}
\end{align}
and similarly for $f^Z$. Accordingly, we can learn a common latent
space\footnote{As we will see in the next section, we actually learn a
  common \emph{distribution} of latent points.}
%, giving us a set of  latent points (mean of the distribution) and associated variance.}
but we allow the two sets of ARD weights $\bfw^Y = \{ w_q^Y
\}_{q=1}^Q$ and $\bfw^Z = \{ w_q^Z \}_{q=1}^Q$ to automatically infer
the responsibility of each latent dimension for generating points in
the $Y$ and $Z$ spaces respectively.  We can then automatically
recover a segmentation of the latent space $X = \left( X^Y, X^s, X^Z
\right)$, where $X^s \in \mathbb{R}^{N \times Q_s}$ is the shared
subspace, defined by the set of dimensions $q \in [1, ... ,Q]$ for
which $w^Y_q, w^Z_q > \delta$, with $\delta$ being a number close
to zero and $Q_s \leq Q$. This equips the model with further
flexibility, because it allows for a ``softly'' shared latent space,
if the two sets of weights are both greater than $\delta$ but
dissimilar, in general.  As for the two private spaces, $X^Y$ and
$X^Z$, they are also being inferred automatically along with their
dimensionalities $Q_Y$ and $Q_Z$ \footnote{In general, there will also
  be dimensions of the initial latent space which are considered
  unnecessary by both sets of weights. 
  %If the subspace
  %corresponding to these irrelevant latent dimensions is denoted with
  %$X^U$, then the actual factorisation can be written more precisely
  %as $X = \left( X^Y, X^s, X^Z, X^U \right)$.
  }.
%
More precisely:
\begin{equation}
  X^Y = \{ \bfx_q \}_{q=1}^{Q_Y}: \bfx_q \in X, \; w^Y_q > \delta, \;  w^Z_q < \delta
\end{equation}
and analogously for $X^Z$. Here, $\bfx_q$ denotes columns of $X$,
while we assume that data are stored by rows.  All of the above are
summarised in the graphical model of figure \ref{fig:grModel}.
%


\subsection{Bayesian training}
\par 
%From the previous section it follows that the model hyperparameters
%are $\bftheta = \{ \sigma_{\epsilon}^{\{Y,Z\}}, \sigma_{ard}^{\{Y,Z\}}, \bfw^{\{Y,Z\}} \}$.
The fully Bayesian training procedure requires maximisation of
the logarithm of the joint \emph{marginal} likelihood $p(Y,Z |
\bftheta) = \int p(Y,Z | X, \bftheta) p(X) \intd X$, where a prior
distribution is placed on $X$. This prior may be a standard normal distribution or may generally depend on a set
of parameters $\bftheta^X$. By
looking again at \eqref{eq:likelihood} we see that the above integral is
intractable due to the nonlinear way in which $X$ appears in
$p(F^{\{Y,Z\}}|X,\bftheta^{\{Y,Z\}})$. Standard variational approximations
are also intractable in this situation. Here, we describe a non-standard
method which leads to an analytic solution.

%Similarly to standard mean field approximations, we start by seeking to maximise
As a starting point, we consider the mean field methodology and seek to maximise
a variational lower bound $F_v(q,\bftheta)$ on the logarithm of the
true marginal likelihood by relying on a variational distribution which factorises as
$q(\Theta)q(X)$, where we assume that $q(X)
\sim \mathcal{N}(\bfmu, S)$. 
%As will be explained later more clearly, in our approach 
As will be explained later more clearly, in our approach $q(\Theta)$ is a distribution which depends on additional
variational parameters $\Theta=\{\Theta^Y, \Theta^Z\}$ so that $q(\Theta) = q(\Theta^Y) q(\Theta^Z)$.
These additional parameters $\Theta$ as well as the exact form of $q(\Theta)$
will be defined later on, as they constitute the most crucial ingredient of our non-standard variational approach.
%$ = q(\Theta^Y)q(\Theta^Z)$ is a 
% and $q(\Theta)$ depends on additional variational
%parameters $\Theta$ and will be defined later on,
%as its form is the main ingredient of our non-standard variational
%approach.

  By dropping the model hyperparameters $\bftheta$ from our expressions, 
  for simplicity, we can use Jensen's inequality and obtain a variational
bound $F_v(q) \leq \log p(Y,Z)$:
\begin{align}
F_v(q) & = \int q(\Theta) q(X) \log \left( \frac{p(Y|X)p(Z|X)}{q(\Theta)}\frac{p(X)}{q(X)} \right) \intd X \nonumber \\
      % & = \int q(\Theta) q(X) \log \frac{p(Y,Z|X)}{q(\Theta)} \intd X - \int \cancel{q(\Theta)} q(X) \log \frac{q(X)}{p(X)} \intd X
      & = \mathcal{L}_Y + \mathcal{L}_Z - \text{KL} \left[ q(X) \parallel p(X)\right], \label{eq:bound}
\end{align}
where % we introduced the additional factorisation $q(\Theta)=q(\Theta^Y)q(\Theta^Z)$ so that
$\mathcal{L}_Y = \int q(\Theta^Y) q(X) \log
\frac{p(Y|X)}{q(\Theta^Y)} \intd X$ and similarly for $\mathcal{L}_Z$.
However, this does not solve the problem of intractability since the
challenging terms still appear in $\mathcal{L}_Y$ and $\mathcal{L}_Z$.
To circumvent this problem, we follow \citet{Titsias:bayesGPLVM10} and
apply the ``data augmentation'' principle, \ie we expand the joint
probability space with $M$ extra samples $U^Y$ and $U^Z$ of the latent
functions $f^Y$ and $f^Z$ evaluated at a set of pseudo-inputs (known
as ``inducing points'') $\bar{X}^Y$ and $\bar{X}^Z$
respectively. Here, $U^Y \in \mathbb{R}^{M_Y \times D_Y}$, $U^Z \in
\mathbb{R}^{M_Z \times D_Z}$, $\bar{X}^Y \in \mathbb{R}^{M_Y \times
  Q}$, $\bar{X}^Z \in \mathbb{R}^{M_Z \times Q}$ and $M = M_Y+M_Z$.
The expression of the joint probability is as before except for the
term $p(Y|X)$ which now becomes:
 \begin{align}
  p(Y|X, \bar{X}^Y) = \int & p(Y|F^Y)p(F^Y|U^Y,X, \bar{X}^Y) \cdot \nonumber \\
  				           & p(U^Y|\bar{X}^Y) \intd F^Y \intd U^Y \label{eq:pYXXbar}
  \end{align}
%
and similarly for $p(Z|X)$. 
The integrations over $U^{\{Y,Z\}}$ are tractable if we assume 
Gaussian prior distributions for these variables.
%
As we shall see, the inducing points are \emph{variational} rather than
model parameters.  More details on the variational learning of
inducing variables in GPs can be found in
\citet{Titsias:variational09}.
%
\par Analogously to \citet{Titsias:bayesGPLVM10}, we are now able to
define $q(\Theta) =  q(\Theta^Y) q(\Theta^Z) $ as
\begin{equation}
\label{qTheta}
q(\Theta) = \prod_{\mathcal{K}=\{Y,Z\}} q(U^\mathcal{K}) p(F^\mathcal{K}|U^\mathcal{K},X,\bar{X}^\mathcal{K}),
\end{equation} 
where $q(U^{\{Y,Z\}})$ are free form distributions.  In that way, the
$p(F^\mathcal{K}|U^\mathcal{K},X,\bar{X}^\mathcal{K})$ factors cancel
out with the ``difficult'' terms of $\mathcal{L}_Y $ and
$\mathcal{L}_Z$, as can be seen by replacing equations \eqref{qTheta}
and \eqref{eq:pYXXbar} back to \eqref{eq:bound}, which now becomes our
final objective function and can be trivially extended for more than
two observed datasets. This function is jointly maximised with respect
to the model parameters, involving the latent space weights $\bfw^Y$
and $\bfw^Z$, and the variational parameters $\{\bfmu, S,
\bar{X}\}$. As in standard variational inference, this optimisation
gives, as a by-product, an approximation of $p(X|Y,Z)$ by $q(X)$, \ie
we obtain a distribution over the latent space. This adds extra
robustness to our model, since previous approaches rely on MAP
estimates for the latent points. More detailed derivation of the
variational bound can be found in the suppl.\ material.


%\subsection{Dynamical Modelling}
\noindent{\bf Dynamical Modelling:}
The model formulation described previously is also
covering the case when we wish to additionally model correlations
between datapoints of the same output space, \eg when $Y$ and $Z$ are
multivariate timeseries. For the dynamical scenario we follow
\citet{Damianou:vgpds11,Lawrence:hgplvm07} and choose the prior on the
latent space to depend on the observation times $\bft \in \mathbb{R}^N$, \eg a GP with a covariance function $k =
k(t,t^\prime)$. With this approach, we are also allowed to learn the
structure of multiple independent sequences which share some commonality by
learning a common latent space for all timeseries while, at the same
time, ignoring correlations between datapoints belonging to
different sequences.





%\subsection{Inference \label{inference}}
\noindent{\bf Inference:}
Given a model which is trained to jointly represent two output spaces
$Y$ and $Z$ with a common but factorised input space $X$, we wish to
generate a new (or infer a training) set of outputs $Z^* \in
\mathbb{R}^{N^* \times D_Z}$ given a set of (potentially partially)
observed test points $Y^* \in \mathbb{R}^{N^* \times D_Y}$.  This is
done in three steps. Firstly, we predict the set of latent points $X^*
\in \mathbb{R}^{N^* \times Q}$ which is most likely to have generated
$Y^*$. For this, we use an approximation to the posterior
$p(X^*|Y^*,Y)$, which has the same form as for the standard Bayesian
GP-LVM model \cite{Titsias:bayesGPLVM10} and is given by a variational
distribution $q(X,X^*)$. To find $q(X,X^*)$ we optimise a variational
lower bound on the marginal likelihood $p(Y,Y^*)$ which has analogous
form with the training objective function
\eqref{eq:bound}. Specifically, we ignore $Z$ and replace $Y$ with
$(Y,Y^*)$ and $X$ with $(X,X^*)$ in \eqref{eq:bound}.
%similar to the variational distribution $q(X)$ learned during training.
In the second step, we find the training latent points $X_{NN}$ which
are closest to $X^*$ in the \emph{shared} latent space.  In the third
step, we find outputs $Z$ from the likelihood $p(Z | X_{NN})$.  This
procedure returns the set of training points $Z$ which best match the
observed test points $Y^*$.  If we wish to generate novel outputs, we
have to propagate the information recovered when predicting $X^*$.
Since the shared latent space encodes the same kind of information for
both datasets, we can achieve the above by simply replacing the
features of $X_{NN}$ corresponding to the shared latent space, with
those of $X^*$.
%
 % A simple way of doing this is to replace the features of $X_{NN}$
 % corresponding to the shared latent space, with those of $X^*$.
 % This is a reasonable idea, since the shared latent space encodes
 % the same kind of information for both datasets.


% A slightly more sophisticated approach is to also exploit the
% continuous nature of the optimised weights $\bfw_Y$ and give less
% importance to the dimensions of $X^*$ for which $w_Y^q$ is small,
% because these features are predicted with large uncertainty
% (variance).  For example, we can create a new set of latent points
% $\hat{X}^{*}$ so that its private dimensions match those of $X_{NN}$
% and its shared dimensions are found by averaging the shared
% dimensions of $X^*$ and $X_{NN}$ as appropriate based on
% $\bfw_Y$. We can then generate outputs from $p(Z^* | \hat{X}^{*})$.


%\subsection{Complexity}
\noindent{\bf Complexity:}
As in common sparse methods in Gaussian processes
\cite{Titsias:variational09}, the typical cubic complexity reduces to
$O(NM^2)$, where $N$ and $M$ is the total number of training and
inducing points respectively. 
In our experiments we set $M=100$.
 Further, the model scales only
linearly with the data dimensionality. Indeed, the Gaussian densities
in equation \eqref{eq:bound} result in an objective function which
only involves the data matrices $Y$ and $Z$ in expressions of the form
$Y Y^\T$ and $Z Z^\T$ which are $N \times N$ matrices no matter how
many features $D_Y$ and $D_Z$ are used to describe the original
data. Also, these quantities are constant and can be
precomputed. Consequently, our approach can model datasets with
very large numbers of features.


%%% Local Variables: 
%%% mode: latex
%%% TeX-master: "../svargplvmICML2012"
%%% End: 


\section{Experiments \label{experiments}}
The MRD method is designed to represent multiple views of a data set
as a set of factorized latent spaces. In this section we will show
experiments which exploit this factorized structure. Source code for
recreating these experiments is included as supplementary material.

%learn the shared and private spaces 
%of distinct datasets which, nevertheless, have some underlying commonality.
%For this reason, for each experiment we consider \emph{pairs}
% of (possibly heterogeneous) datasets, although the model can as well be applied
%to more than two subsets.
%The various properties of the model are explored by considering different tasks (visualisation, datapoint correspondence,
%data generation, classification) and datasets that are different in nature.
%
%The three pairs of datasets considered here are different in nature, 
%allowing us to explore the various properties of our method. 
%
%Firstly,
%we consider a set of very high dimensional images belonging two six different human faces, spread into two datasets. The principal underlying
%commonality of the two subsets, which our method effectively discovers, is the varying light condition of the images. As a second
%experiment, we opted for a set of recordings of various human motions which is provided in two different representations: a subset containing %pose data and a subset containing the
%the corresponding silhouette features. 
%
%The performance of the model is evaluated in different tasks, such as visualisation and interpretation
%of the latent space which is discovered and segmented automatically, correspondence of datapoints between the two subsets of the given
%datasets, as well as generation of new data and classification.
%
%Source code for recreating these experiments is included as supplementary material.

%\subsection{Yale faces dataset}
\noindent{\bf Yale faces:}
To show the ability of our method to model very
high-dimensional spaces our first experiment is applied to the Yale
dataset \cite{YaleFaces1, YaleFaces2} which contains images of several
human faces under different poses and 64 illumination conditions.  We
consider a single pose for each subject such that the only variations
are the location of the light source and the subject's appearance.
Since our model is capable of working with very high-dimensional data, %. This means that
it can be directly applied to the raw pixel values (in this
case $192 \times 168 = 32,256$ pixels/image) so that we do not
have to rely on image feature extraction to pre-process the data, and we can directly sample novel outputs.
%% to
%% demonstrate the ability of our method to model data with a very large
%% number of features.  With this approach, we can also directly sample
%% new images from the learned model.
% \subsubsection{Modeling one face}
%Before we proceed to subspace modelling, we first fit the standard Bayesian GP-LVM model to the whole set of $64$ images belonging to a single face, to visualise and assess the quality of the discovered latent space.
%The model was initialised with $Q=15$ latent dimensions, and the Bayesian training not only discovered the effective dimensionality of the latent space automatically,
%but it also defined the ``importance'' of each dimension. As can be seen in figure \ref{fig:yaleOneFaceScales}, most of the mapping kernel's weights were driven close to zero, signifying that the 
%latent space is dominated by the three dimensions which have been assigned large weights.
%
%In figures
%\ref{fig:yaleOneFaceX21} and \ref{fig:yaleOneFaceX23}, one can see that the projection of the latent space into the 3 most dominant dimensions is shaped as a hollow 
%hemishpere, which is in accordance with the shape of the space defined by the fixed locations of the light source.
%
%\hspace{-6pt}
%\begin{figure}[ht]
%\begin{center}
%\subfigure[]{
%\includegraphics[width=0.145\textwidth]{../diagrams/Yale1Face/scales}
%	\label{fig:yaleOneFaceScales}
%}
%\hspace{-5pt}
%\subfigure[]{
%	\includegraphics[width=0.145\textwidth]{../diagrams/Yale1Face/X21}
%	\label{fig:yaleOneFaceX21}
%}
%\hspace{-5pt}
%\subfigure[]{
%	\includegraphics[width=0.145\textwidth]{../diagrams/Yale1Face/X23}
%	\label{fig:yaleOneFaceX23}
%}
%\end{center}
%\vspace{-7pt}
%\caption{\small{ \it
%The weight set $\bfw$ associated with the learned latent space is shown in \subref{fig:yaleOneFaceScales}.
%In figures \subref{fig:yaleOneFaceX21} and \subref{fig:yaleOneFaceX23} we plotted pairs of the $3$ most dominant
%latent dimensions against each other (dimension 1 against 2 and 1 against 3 respectively).
%}
%}
%\label{fig:yaleOneFace1}
%\vspace{-8pt}
%\end{figure}
%\hspace{-6pt}
%
%This suggests that latent feature indices $1,2$ and $3$ encode the information about the illumination condition.
%Figure \ref{fig:yaleOneFaceScales} also shows that the latent dimensions with indices $4,5$ and $6$ have a very small but not negligible weight.
%%These are retained by the model because
%%apart from the illumination condition, there exist other minor differences among pictures of a single face, as the depicted %persons
%These represent other minor differences between an individual's face pictures, as the subjects
%often blink, slightly move or smile etc. 
%
% \subsubsection{Shared latent spaces for multiple faces}
%In this section we use our model, from now on referred to as \emph{Manifold Relevance Determination model (MRD)} for latent subspace learning. 
%
%To test our model in the Yale face data
From the full Yale database, we constructed a dataset $Y$ containing the pictures corresponding to all $64$ different illumination conditions for each one of $3$
subjects and
%\highlight{Need to be specific about what's in Y and what's in Z,  it's currently not clear!!!}
similarly for $Z$, for $3$ different subjects.
  In this way, we formed two datasets, $Y$ and $Z$,
each consisting of all $64 \times 3$ images corresponding to a set of
three different faces, under all possible illumination conditions, therefore, $Y, Z \in \mathbb{R}^{N \times D}$,
$N = 192$, $D = 32,256$.
 We then aligned the order of the images in each dataset, so that each
image $\bfy_n$ from the first one was randomly set to correspond to one of
the $3$ possible $\bfzi_n$'s of the second dataset which are depicted in the same
illumination condition as $\bfy_n$.
In that way, we matched datapoints between the two datasets only according to the 
illumination condition and not the identity of the faces, so that
the model is not explicitly forced to learn the correspondence between face characteristics.

\par The latent space variational means were initialised by
concatenating the two datasets and performing PCA. An alternative
approach would be to perform PCA on each dataset separately and then
concatenate the two low dimensional representations to initialise $X$.
We found that both initializations achieved similar results. 
The optimized relevance weights $\{\bfw^Y, \bfw^Z\}$
%weights were optimized by maximizing the variational lower bound and 
are
visualized as bar graphs in figure \ref{fig:yale6SetsScales}.
% by the parameters of the generative
%mappings $f^Y$ and $f^Z$ as shown in figure \ref{fig:yale6SetsScales}.

\begin{figure}[ht]
\begin{center}
\subfigure[]{
\includegraphics[width=0.2\textwidth]{../diagrams/Yale6Sets/scalesMod1_noNum}
	\label{fig:yale6SetsScales1}
}
\subfigure[]{
	\includegraphics[width=0.2\textwidth]{../diagrams/Yale6Sets/scalesMod2_noNum}
	\label{fig:yale6SetsScales2}
}
\end{center}
\vspace{-5pt}
\caption{
%\small{ \it
The relevance  weights for the faces data. Despite allowing for soft sharing, the first 3 dimensions are switched on with approximately the same weight for both views of the data. Most of the remaining dimensions are used to explain private variance. %Dimension 9 is switched off.
% in \subref{fig:yale6SetsScales1} and \subref{fig:yale6SetsScales2},
% define a partitioning and ``soft'' sharing of the latent space. 
%}
}
\label{fig:yale6SetsScales}
%\vspace{-8pt}
\end{figure}

The latent space is clearly segmented into a shared part, consisting
of dimensions indexed as $1$,$2$ and $3$ \footnote{Dimension 6
also encodes shared information, but of almost negligible amount ($w^Y_6$ and $w^Z_6$ are almost zero).}
% is also
%  in the shared set but both models assigned a very small weight, as
%  it encodes an almost negligible amount of information.} 
two private and an irrelevant part (dimension $9$).
%, consisting of dimensions indexed as $\{5,7,11,14 \}$
%and $\{4,8,10,12,13 \}$ respectively.
%  The $9$th feature of every
%latent point was found to be unnecessary for the generation of both
%output spaces. 
The two data views allocated approximately equal
weights to the shared latent dimensions, which %. These three latent dimensions
are visualized in figures
\ref{fig:yale6SetsLatentSpace}\subref{fig:yale6SetsX12} and
\ref{fig:yale6SetsLatentSpace}\subref{fig:yale6SetsX13}. Interaction
with these three latent dimensions reveals that the structure of the
shared subspace resembles a hollow hemisphere. This corresponds to the
shape of the space defined by the fixed locations of the light source.
%

%\hspace{-6pt}
\begin{figure}[ht]
\begin{center}
\subfigure[]{
\includegraphics[width=0.145\textwidth]{../diagrams/Yale6Sets/mod1X_1_2}
	\label{fig:yale6SetsX12}
}
\hspace{-5pt}
\subfigure[]{
	\includegraphics[width=0.145\textwidth]{../diagrams/Yale6Sets/mod1X_1_3}
	\label{fig:yale6SetsX13}
}
\hspace{-5pt}
\subfigure[]{
	\includegraphics[width=0.145\textwidth]{../diagrams/Yale6Sets/mod1_X5_14}
	\label{fig:yale6SetsX5_14}
}
\end{center}
\vspace{-4pt}
\caption{
%\small{ \it
Projection of the shared latent space into dimensions $\{1,2\}$ and $\{1,3\}$ (figures
\subref{fig:yale6SetsX12} and \subref{fig:yale6SetsX13}) and projection of the $Y-$private dimensions $\{5,14\}$
(figure \subref{fig:yale6SetsX5_14}).
%which disambiguate between different faces in the first dataset.
 It is clear how the latent points in figure
\subref{fig:yale6SetsX5_14} form three clusters, each responsible for modelling one of the three faces in $Y$.
%}
%The grey scale background indicates the precision with which the mapping is expressed.
}
\label{fig:yale6SetsLatentSpace}
%\vspace{-8pt}
\end{figure}
%\hspace{-6pt}
%
%the shape of the shared latent space is similar to
%the one found by the Bayesian GP-LVM (figure \ref{fig:yaleOneFace1}), as can be seen in figures \ref{fig:yale6SetsLatentSpace}\subref{fig:yale6SetsX12} and
%\ref{fig:yale6SetsLatentSpace}\subref{fig:yale6SetsX13}.
%
This indicates that the shared space successfully encodes the
information about the position of the light source and not the face
characteristics.
%as the random alignment that we performed forbade the algorithm from modelling commonalities between faces of the
%two different datasets.
%
This indication is enhanced by the results found when we performed
dimensionality reduction with the standard Bayesian GP-LVM for
pictures corresponding to all illumination conditions of a single face (\ie a dataset with one
modality).  Specifically, the latent space discovered by the Bayesian
GP-LVM and the shared subspace discovered by MRD have the same
dimensionality and similar structure, as can be seen in figure
\ref{fig:yaleOneFace1}.
%
\hspace{-6pt}
\begin{figure}[ht]
\begin{center}
\subfigure[]{
\includegraphics[width=0.142\textwidth]{../diagrams/Yale1Face/scalesVargplvm_noNum}
	\label{fig:yaleOneFaceScales}
}
\hspace{-5pt}
\subfigure[]{
	\includegraphics[width=0.145\textwidth]{../diagrams/Yale1Face/X21}
	\label{fig:yaleOneFaceX21}
}
\hspace{-5pt}
\subfigure[]{
	\includegraphics[width=0.145\textwidth]{../diagrams/Yale1Face/X23}
	\label{fig:yaleOneFaceX23}
}
\end{center}
\vspace{-4pt}
\caption{
%\small{ \it
Latent space learned by the standard Bayesian GP-LVM for a single face dataset.
The weight set $\bfw$ associated with the learned latent space is shown in \subref{fig:yaleOneFaceScales}.
In figures \subref{fig:yaleOneFaceX21} and \subref{fig:yaleOneFaceX23} we plotted pairs of the $3$ dominant
latent dimensions against each other. Dimensions $4,5$ and $6$ have a very small but not negligible weight and
represent other minor differences between pictures of the same face, as the subjects often blink, smile etc.
%}
}
\label{fig:yaleOneFace1}
%\vspace{-8pt}
\end{figure}
\hspace{-6pt}
%
\par As for the private manifolds discovered by MRD, these correspond
to subspaces for disambiguating between faces of the same dataset.
Indeed, plotting the largest two dimensions of the first latent
private subspace against each other reveals three clusters,
corresponding to the three different faces within the dataset.
Similarly to the standard Bayesian GP-LVM applied to a single face,
here the private dimensions with very small weight model slight changes
across faces of the same subject (blinking etc).
%
%here the private dimensions with the very small weight are the ones
%that model the minor differences introduced by the fact that the
%subject characteristics are slightly changed in several photos (due to
%blinking etc).
%
%The MRD method not only finds a very efficient and intuitive
%factorization of the latent space into lighting conditions and facial
%representation, but this factorization is also found automatically through the
%approximate variational Bayesian algorithm.

We can also confirm visually the subspaces' properties by sampling a
set of novel inputs $X_{samp}$ from each subspace and then mapping
back to the observed data space using the likelihoods $p(Y|X_{samp})$ or $p(Z|X_{samp})$,
thus obtaining novel outputs (images).
%Being able to do so is an important advantage of our method (which is generative), because it also means that we can generate novel datapoints from the trained model.
%The intuitive segmentation of the latent space also helps towards this direction. 
To better understand what kind of information is encoded in each of
the dimensions of the shared or private spaces, we sampled new latent
points by varying only one dimension at a time, while keeping the rest
fixed.
%
The first two rows of figure \ref{fig:yale6SetsInterpolation} show
some of the outputs obtained after sampling across each of the shared
dimensions $1$ and $3$ respectively, which clearly encode the
coordinates of the light source, whereas dimension $2$ was found to
model the overall brightness. The sampling procedure can intuitively
be thought as a walk in the space shown in figure
\ref{fig:yale6SetsLatentSpace}\subref{fig:yale6SetsX13} from left to
right and from the bottom to the top. Although the set of learned
latent inputs is discrete, the corresponding latent subspace is
continuous, and we can interpolate images in new illumination
conditions by sampling from areas where there are no training inputs
(\ie in between the red crosses shown in figure
\ref{fig:yale6SetsLatentSpace}).

% Using a generative model also allows us to sample new datapoints from the trained model. This is done by selecting a training input $\bfx_n$ (\ie one of
% the latent points corresponding to an existing, observed $\bfy_n$), and sampling one or more of its dimensions from the whole of the corresponding feature space,
% obtaining, thus, a novel input $\bfx_*$. We can then obtain novel outputs by using the likelihood $p(\bfy_n | \bfx_n)$. Figures \ref{}, \ref{} and \ref{} show some of the outputs obtained after
% sampling across each of the principle dimensions by hand. When the sampled $\bfx_*$ were close or exactly similar to training inputs, the corresponding output
% looks like one of the training datapoints, otherwise the output is novel. From the aforementioned figures, one can see that the two of the principal dimensions
% represent the change of the light source location along the $X$ and $Y$ axis respectively, whereas the third one is responsible for modelling the brightness.

Similarly, we can sample from the private subspaces and obtain novel
outputs which interpolate the non-shared characteristics of the
involved data.  This results in a morphing effect across different
faces, which is shown in the last row of figure
\ref{fig:yale6SetsInterpolation}.  Example videos can be
found in the supplementary material.


\begin{figure*}[ht]
\begin{center}
\subfigure{ \includegraphics[width=0.09\textwidth]{../diagrams/Yale6Sets/lightInterpolation/X13_1000} }
\subfigure{ \includegraphics[width=0.09\textwidth]{../diagrams/Yale6Sets/lightInterpolation/X13_1009} }
\subfigure{ \includegraphics[width=0.09\textwidth]{../diagrams/Yale6Sets/lightInterpolation/X13_1021} }
\subfigure{ \includegraphics[width=0.09\textwidth]{../diagrams/Yale6Sets/lightInterpolation/X13_1022} }
\subfigure{ \includegraphics[width=0.09\textwidth]{../diagrams/Yale6Sets/lightInterpolation/X13_1036} }
\subfigure{ \includegraphics[width=0.09\textwidth]{../diagrams/Yale6Sets/lightInterpolation/X13_1040} }
\subfigure{ \includegraphics[width=0.09\textwidth]{../diagrams/Yale6Sets/lightInterpolation/X13_1046} }
\subfigure{ \includegraphics[width=0.09\textwidth]{../diagrams/Yale6Sets/lightInterpolation/X13_1055} }
\subfigure{ \includegraphics[width=0.09\textwidth]{../diagrams/Yale6Sets/lightInterpolation/X13_1063} }
\vspace{-8pt}
\newline
\subfigure{ \includegraphics[width=0.09\textwidth]{../diagrams/Yale6Sets/lightInterpolation/X13_1064} }
\subfigure{ \includegraphics[width=0.09\textwidth]{../diagrams/Yale6Sets/lightInterpolation/X13_1072} }
\subfigure{ \includegraphics[width=0.09\textwidth]{../diagrams/Yale6Sets/lightInterpolation/X13_1079} }
\subfigure{ \includegraphics[width=0.09\textwidth]{../diagrams/Yale6Sets/lightInterpolation/X13_1085} }
\subfigure{ \includegraphics[width=0.09\textwidth]{../diagrams/Yale6Sets/lightInterpolation/X13_1095} }
\subfigure{ \includegraphics[width=0.09\textwidth]{../diagrams/Yale6Sets/lightInterpolation/X13_1110} }
\subfigure{ \includegraphics[width=0.09\textwidth]{../diagrams/Yale6Sets/lightInterpolation/X13_1125} }
\subfigure{ \includegraphics[width=0.09\textwidth]{../diagrams/Yale6Sets/lightInterpolation/X13_1137} }
\subfigure{ \includegraphics[width=0.09\textwidth]{../diagrams/Yale6Sets/lightInterpolation/X13_1149} }
\vspace{-8pt}
\newline
\subfigure{ \includegraphics[width=0.1\textwidth]{../diagrams/Yale6Sets/morphing/1054} }
\subfigure{ \includegraphics[width=0.1\textwidth]{../diagrams/Yale6Sets/morphing/1079} }
\subfigure{ \includegraphics[width=0.1\textwidth]{../diagrams/Yale6Sets/morphing/1089} }
\subfigure{ \includegraphics[width=0.1\textwidth]{../diagrams/Yale6Sets/morphing/1094} }
\subfigure{ \includegraphics[width=0.1\textwidth]{../diagrams/Yale6Sets/morphing/1102} }
\subfigure{ \includegraphics[width=0.1\textwidth]{../diagrams/Yale6Sets/morphing/1106} }
\subfigure{ \includegraphics[width=0.1\textwidth]{../diagrams/Yale6Sets/morphing/1123} }
\end{center}
%\vspace{-4pt}
\caption{
%\small{ \it
Sampling inputs to produce novel outputs.
First row shows interpolation between positions of the light source in the $x$ coordinate
and second row in the $y$ coordinate (elevation). Last row shows interpolation between
face characteristics to produce a morphing effect. Note that these images are presented scaled here, see suppl. material for the original $32,256$-dimensional ones.
%}
}
\label{fig:yale6SetsInterpolation}
%\vspace{-8pt}
\end{figure*}


\par As a final test, we confirm the efficient segmentation of the
latent space into private and shared parts by automatically recovering
all the illumination similarities found in the training set.  More
specifically, given a datapoint $\bfy_n$ from the first dataset, we
search the whole space of training inputs $X$ to find the $6$ Nearest
Neigbours to the latent representation $\bfx_n$ of $\bfy_n$, based
only on the shared dimensions.
%In other words, we compare $x_{n,i}$ to all the rest $\{x_{n,j}\}_{n=1}^N$, where $i \neq j$ and $i,j$ belong to the set
% of the shared dimensions. 
 From these latent points, we can then obtain points in the output
 space of the second dataset, by using the likelihood $p(Z | X)$.  As
 can be seen in figure \ref{fig:yale6SetsGrouping}, the model returns
 images with matching illumination condition.  Moreover, the fact
 that, typically, the first neighbours of each given point correspond
 to outputs belonging to different faces, indicates that the shared
 latent space is ``pure'', and is not polluted by information that
 encodes the face appearance.
%\hspace{-6pt}
\begin{figure}[ht]
\begin{center}
\subfigure{ \includegraphics[width=0.47\textwidth]{../diagrams/Yale6Sets/grouping/givenMod2/122} }
 \vspace{-16pt}
 \newline
\subfigure{ \includegraphics[width=0.47\textwidth]{../diagrams/Yale6Sets/grouping/givenMod2/107} }
 \vspace{-16pt}
 \newline
\subfigure{ \includegraphics[width=0.47\textwidth]{../diagrams/Yale6Sets/grouping/givenMod1/24} }
 \vspace{-16pt}
 \newline
\subfigure{ \includegraphics[width=0.47\textwidth]{../diagrams/Yale6Sets/grouping/givenMod1/70} }
 \vspace{-16pt}
 \newline
\end{center}
\vspace{-7pt}
\caption{
%\small{ \it
Given the images of the first column, the model searches only in the shared latent space to find the pictures of the opposite dataset
which have the same illumination condition. The images found, are sorted in columns
$2$ - $7$ by relevance.
%}
}
\label{fig:yale6SetsGrouping}
\vspace{-8pt}
\end{figure}

%\hspace{-6pt}

%%%% THE FIRST NN is the same latentn point, we should look from NN2 and so on.


%\subsection{Human motion data}
\noindent{\bf Human motion data:}
For our second experiment, we consider a set of $3$D human poses and
associated silhouettes, coming from the dataset of Agarwal and Triggs
\cite{Agarwal:pose06}. We used a subset of $5$ sequences, totalling
$649$ frames, corresponding to walking motions in various directions
and patterns.  A separate walking sequence of $158$ frames was used as
a test set.  Each pose is represented by a $63-$dimensional vector of
joint locations and each silhouette is represented by a
$100-$dimensional vector of HoG (histogram of oriented gradients) features.

\par Given the test silhouette features, we used our model to generate
the corresponding poses. This is challenging, as the data
are multi-modal, \ie a silhouette representation may be generated from
more than one poses (\eg fig. \ref{fig:humanPoseAmbiguity}).

\begin{figure}[ht]
\begin{center}
%  \includegraphics[width=0.43\textwidth]{../diagrams/humanPose/ambiguity2FinalCombined3}
  \includegraphics[width=0.3\textwidth]{../diagrams/humanPose/newAmbiguity}
\end{center}
\vspace{-7pt}
\caption{
%\small{ \it
Although the two poses in the second column are very dissimilar, they correspond to resembling silhouettes
that have similar feature vectors. This happens because the $3$D information is lost in the silhouette space,
as can also be seen in the third column, depicting the same poses from the silhouettes' viewpoint.
%}
}
\label{fig:humanPoseAmbiguity}
%\vspace{-4pt}
\end{figure}

As described in the inference section, given $\bfy^*$, one of the
$N^*$ test silhouettes, our model optimises a test latent point
$\bfx^*$ and finds a series of $K$ candidate initial training inputs $
\{ \bfx_{NN}^{(k)} \}_{k=1}^K$, sorted according to their similarity
to $\bfx^*$, taking into account only the shared dimensions.  Based on
these initial latent points, it then generates a sorted series of $K$ \emph{novel}
poses $\{ \bfzi^{(k)} \}_{k=1}^K$. For the dynamical version of our
model, all test points are considered together and the predicted
$N^*$ outputs are forced to form a smooth sequence.  Our experiments
show that the initial training inputs $\bfx_{NN}$ typically
correspond to silhouettes similar to the given one, something which
confirms that the segmentation of the latent space is
efficient. However, when ambiguities arise, as the example shown in
figure \ref{fig:humanPoseAmbiguity}, the non-dynamical version of our
model has no way of selecting the correct input, since all points of the
test sequence are treated independently. But when the dynamical
version is employed, the model forces the whole set of training and
test inputs to create smooth paths in the latent space.
 %, as can be seen in figure \ref{fig:humanPoseLatentSpaces}. 
 In other words, the dynamics disambiguate the model.  
%
%\begin{figure}[ht]
%\begin{center}
%\subfigure[]{ \includegraphics[width=0.18\textwidth]{../diagrams/humanPose/latentSpaceStatic} 
%\label{fig:latentSpaceStatic}
%} \hspace{-3pt}
%\subfigure[]{ \includegraphics[width=0.18\textwidth]{../diagrams/humanPose/latentSpaceDynCropped} 
%\label{fig:latentSpaceDyn}
%}
%\end{center}
%\vspace{-9pt}
%\caption{\small{ \it Projection of the latent space discovered for the non-dynamical model \subref{fig:latentSpaceStatic}
%and for the dynamical model \subref{fig:latentSpaceDyn} onto their two principal dimensions.
%}
%}
%\label{fig:humanPoseLatentSpaces}
%\vspace{-3pt}
%\end{figure}


 Indeed, as can be seen in figure \ref{fig:humanPoseAmbiguityTest},
 our method is forced to select a candidate training input $\bfx_{NN}$
 for initialisation which does not necessarily correspond to the
 training silhouette that is most similar to the test one.
%
%In other words, the predicted pose is generated by a latent point which is not
%necessarily initialised in the latent point that is found by Nearest Neighbour in the silhouette space. 
%
 What is more, if we assume that the test \emph{pose} $\bfzi^*$ is
 known and seek for its nearest training neighbour in the pose space,
 we find that the corresponding silhouette is very similar to the one
 found by our model, which is only given information in the silhouette
 space.

\begin{figure}[ht]
\begin{center}
  \includegraphics[width=0.35\textwidth]{../diagrams/humanPose/ambiguityTest}
\end{center}
\caption{
%\small{\it 
Given the HoG features for the test silhouette in column one, we predict the corresponding pose using the dynamical version of MRD and Nearest Neighbour (NN) in the silhouette space
obtaining the results in the first row, columns 2 and 3 respectively. The last row is the same as the first one, but the poses are rotated
to highlight the ambiguities. Notice that the silhouette shown in the second row for MRD does not correspond exactly to the pose
of the first row, as the model generates only a \emph{novel} pose given a test silhouette. Instead, it is the training silhouette found by performing
NN in the shared latent space. 
The NN of the training \emph{pose} given the test pose is shown in column $4$.
%}
}
\label{fig:humanPoseAmbiguityTest}
\end{figure}

\par Given the above, we quantify the results and compare our method
with linear and Gaussian process regression and Nearest Neighbour in
the silhouette space. We also compared against the shared GP-LVM
\cite{Ek:2008up, Ek:2009vv} which optimises the latent points using
MAP and, therefore, requires an initial factorisation of the inputs to
be given a priori.  Finally, we compared to a dynamical version of Nearest Neighbour where
we kept multiple nearest neighbours and selected the coherent ones over a sequence.
% Specifically,
%for every given test point $\bfy^*_i$ we find the $P$ closest neighbours in the training set and 
%store them in a $P \times D_Y$ matrix $Y^{(nn)}_i$.
%These correspond to a set $Z^{(nn)}_i$ of points in the other view.
%Then, to predict $\bfzi^*_i$ we select the point $\bfzi^{(nn)}_i$ taken from $Z^{(nn)}_i$
% which is closest to $\bfzi^*_{i-1}$, thus forcing an auto-regressive behaviour
%\footnote{We report the error achieved for the best starting pose $\bfzi^*_1$ and the best $P$, taken from a set of possible values $(1,...,10)$.}.
 %
The errors shown in table \ref{humanMotionTable}
as well as the video provided as supplementary material show that MRD
performs better than the other methods in this task.



\begin{table}[h]
\caption{
\label{humanMotionTable}
%\small{ \it
The mean of the Euclidean distances of the joint locations between the predicted and the true poses.
The Nearest Neighbour in the pose space
is not a fair comparison, but is reported here as it provides some insight about the
lower bound on the error that can be achieved for this task.
%}
}
\begin{small}
\begin{center}
\begin{tabular}{ l | l }
%\hline
						 	   & Error \\ \hline %\hline
Mean Training Pose		       & 6.16   \\ %\hline
Linear Regression		       & 5.86   \\ %\hline
GP Regression 			       & 4.27   \\ %\hline
Nearest Neighbour (sil. space) & 4.88  \\ %\hline
Nearest Neighbour with sequences (sil. space) & 4.04  \\ %\hline
\textcolor{light-gray}{Nearest Neighbour (pose space)} & \textcolor{light-gray}{2.08}   \\ %\hline
%Nearest Neighbour (pose space) & 2.08   \\ %\hline %---
Shared GP-LVM				       & 5.13    \\ % \hline
MRD	without Dynamics       & 4.67   \\ %\hline
MRD with Dynamics	       & \textbf{2.94}    \\ \hline
\end{tabular}
\end{center}
\end{small}
\end{table}



%\subsection{Classification}
\noindent{\bf Classification:}
As a final experiment, we demonstrate the flexibility of our model in
a supervised dimensionality reduction scenario for a classification
task. The training dataset was created such that a matrix $Y$
contained the actual observations and a matrix $Z$ the corresponding
class labels in 1-of-K encoding.  We used the `oil' database
\cite{Bishop:oil93} which contains $1000$ $12-$dimensional examples
split in $3$ classes.  We selected $10$ random subsets of the data
with increasing number of training examples and compared to the
nearest neighbor (NN) method in the data space. As can be seen in
figure \ref{fig:oilErrors}, MRD successfully determines the shared
information between the data and the label space and outperforms NN.
%\vspace{-0.2cm}
\begin{figure}[ht]
\begin{center}
  \includegraphics[width=0.4\textwidth]{../diagrams/errorAccuracy}
\end{center}
\vspace{-0.1cm}
\caption{
Accuracy obtained after testing MRD and NN on the full test set of the `oil' dataset.
}
\label{fig:oilErrors}
%\vspace{-0.2cm}
\end{figure}


%%% Local Variables: 
%%% mode: latex
%%% TeX-master: "../svargplvmICML2012"
%%% End: 


\section{Conclusions \label{conclusions}}
%% Many applications in computer vision and related fields are concerned
%% with modelling in scenarios where multiple streams of information of
%% the same underlying phenomenon are available. Further, the data is
%% often very high dimensional with enormous redundancies.
%Such a representation often means that the data
%is distributed on or close to a manifold through the observed
%parametrization.  
We have presented a new factorized latent variable model for multi
view data.  The model automatically factorizes the data using
variables representing variance that exists in each view separately
from variance being specific to a particular view. 
%% that automatically factorizes the latent space into variables that
%% are either shared or specific to one of the views of the
%% data.
% We introduced a relaxation to the discrete segmentation
%% of the latent representation 
% and allow for a ``softly'' shared latent space.
%The model learns the structure of the latent space variationally,
% allowing it to incorporate prior distributions for the latent space. 
The model learns a distribution over the latent points
variationally. This allows us to to automatically find the
dimensionality of the latent space as well as to incorporate prior
knowledge about its structure.
%
As an example, we showed how dynamical priors can be included on the latent
space. This allowed us to use temporal continuity to disambiguate the
model's predictions in an ambiguous human pose estimation problem.
%
%We exploited the factorization to perform human
%pose estimation in an ambiguous setting.
% where the model separates the variance in the pose
%space that can be determined from the image observations from the one
%that is ambiguous. 
% Our model allows for dynamical priors to be
% incorporated when learning it. This allowed us to disambiguate
% in the pose estimation example. 
%
The model is capable of learning from extremely high-dimensional
data. We illustrated this by learning a model directly on the pixel
representation of an image. Our model is capable of learning a compact
an intuitive representation of such data which we exemplified by
generating novel images by sampling from the latent representation in a
structured manner.
%% applying it to images of several different
%% faces under the same set of lighting conditions. Images of faces from
%% novel lighting directions or with novel appearance could be
%% synthesized by sampling from the corresponding latent space.
%
%The model is capable of learning from
%extremely high-dimensional data. By applying it to images of
%several different faces under the same set of lighting conditions the model
%correctly finds the generating low-dimensional parameters of the data
%separated into facial appearance and illumination direction. From the
%resulting model we showed how images of faces from novel lighting
%directions or with novel facial characteristics could be synthesized.
%
%
Finally, we showed how a generative model with discriminative
capabilities can be obtained by treating the observations and class labels
of a dataset as separate modalities.  

% As future work, we envisage
% approaches with more sophisticated ways of directly constraining the
% latent space through priors. 
% %In classification scenarios, for
% %example, we could consider a latent space prior evaluated at the class
% %labels.
%  Further, it would be interesting to explore the possibility of
% incorporating different latent space constraints for each different
% observed modality.


%%% Local Variables: 
%%% mode: latex
%%% TeX-master: "../svargplvmICML2012"
%%% End: 



\section*{Acknowledgments}
Research was partially supported by the University of Sheffield Moody endowment fund and the Greek State Scholarships Foundation (IKY).
We would like to thank the reviewers for their useful feedback.



\small
\bibliography{svargplvmICML2012,other,lawrence,zbooks}
\bibliographystyle{icml2012}
%}

%\newpage
 \begin{center}
 \begin{Large}
 \textbf{
% Gaussian Process Dynamical Systems\\
 Appendix
 } \\
 \end{Large}
% \noindent \newline
% \textbf{Andreas Damianou, Michalis Titsias, Neil Lawrence}
 \end{center}
\appendix
\section{Derivation of the variational bound}

We wish to approximate the marginal likelihood:
\begin{equation}
\label{marginalLikelihoodSuppl}
p(Y | \bft) =  \int p( Y , F, X| \bft) \intd  X \intd F,
\end{equation}
by computing a lower bound:
\begin{align}
\mathcal{F}_v(q, \boldsymbol \theta) = {}& \int q(\mathit{\Theta}) \log 
		\frac{ p(Y , F , \mathit{X} | \mathbf{t})}
			 {q(\mathit{\Theta})}  \intd  X \intd F.
% 	    \nonumber \\
% 	      = {}& \sum_{d=1}^D \int q(\Theta) \log \left( p(\bfy_d | \bff_d) p(\bff_d | X) \right) dX d \bff_d -
% 		    \int q(\Theta) \frac{p(X|\bft)}{q(\Theta)} dX
		 \label{jensens1Suppl}
\end{align}
%
This can be achieved by first augmenting the joint probability density of our model with inducing inputs $\tilde{X}$ along with their corresponding function values $U$:
\begin{equation}
 \label{augmentedJointSuppl}
p(Y,F, U,X,\tilde{X} | \bft) = \prod_{d=1}^D p(\mathbf{y}_d | \mathbf{f}_d) p(\mathbf{f}_d | \mathbf{u}_d, \mathit{X})
p(\bfu_d | \tilde{X})  p(X | \mathbf{t})
\end{equation}
where $p(\bfu_d | \tilde{X}) = \prod_{d=1}^D \mathcal{N} \left( \bfu_d | \mathbf{0}, K_{MM} \right)$ . For simplicity, $\tilde{X}$ is dropped from our
expressions for the rest of this supplementary material. Note that after including the inducing points, $p(\bff_d | \bfu_d, X)$
remains analytically tractable and it turns out to be \cite{rasmussen-williams}):
\begin{equation}
 \label{priorF2Suppl}
p(\bff_d | \bfu_d, X) =  \mathcal{N}  \left( \bff_d | K_{NM} K_{MM}^{-1} \bfu_d , K_{NN} - K_{NM} K_{MM}^{-1} K_{MN} \right).
\end{equation}
For tractability, we now define a variational density $q(\Theta)$:
\begin{equation}
\label{varDistrSuppl}
q(\mathit{\Theta}) = q(F, U,X) = q(F | U, X) q(U) q(X) = \prod_{d=1}^D p(\bff_d | \bfu_d, X )q(\bfu_d) q(X),
\end{equation}
%
%
where $q(X) = \prod_{q=1}^Q \mathcal{N} \left( \bfx_q | \bfmu_q, S_q \right)$. 
%
Now, we return to \eqref{jensens1Suppl} and replace the joint distribution with its augmented version \eqref{augmentedJointSuppl} and the variational distribution with its factorised version \eqref{varDistrSuppl}:
\begin{align}
\mathcal{F}_v(q, \boldsymbol \theta) = {}& \int q(\mathit{\Theta}) \log 
		\frac{ p(Y,F, U,X | \bft)}
			 {q(F, U,X)}  \intd  X \intd F,
 	    \nonumber \\
= {}& \int \prod_{d=1}^D p(\bff_d | \bfu_d, X )q(\bfu_d) q(X) 
	    \log  \frac{\prod_{d=1}^D p(\mathbf{y}_d | \mathbf{f}_d) \cancel{p(\mathbf{f}_d | \mathbf{u}_d, \mathit{X})}
						p(\bfu_d | \tilde{X})  p(X | \mathbf{t})}
 	      		   {\prod_{d=1}^D \cancel{p(\bff_d | \bfu_d, X )}q(\bfu_d) q(X)}   \intd  X \intd F \nonumber \\
= {}& \int \prod_{d=1}^D p(\bff_d | \bfu_d, X )q(\bfu_d) q(X) 
		\log  \frac{\prod_{d=1}^D p(\mathbf{y}_d | \mathbf{f}_d) p(\bfu_d | \tilde{X})}
				   {\prod_{d=1}^D q(\bfu_d) q(X)}   \intd  X \intd F \nonumber \\
- {}& \int \prod_{d=1}^D  q(X)   \log \frac{q(X)}{p(X | \mathbf{t})}   \intd  X \nonumber \\
= {}& \hat{\mathcal{F}}_v - \text{KL}(q \parallel p), \label{jensensSuppl}
\end{align}
%
with:
 \begin{equation}
\hat{\mathcal{F}}_v = 
\sum_{d=1}^D \left( 
    \int q(\bfu_d) q(X) \left\langle \log p(\bfy_d | \bff_d) \right\rangle_{p(\bff_d | \bfu_d, X)} d\bfu_d \; dX +
					   \log \left\langle \frac{p(\bfu_d)}{q(\bfu_d)} \right\rangle_{q(\bfu_d)} 
  \right) = \sum_{d=1}^D \hat{\mathcal{F}}_d
\end{equation} 

Both terms in \eqref{jensensSuppl} are analytically tractable, with the first having the same analytical solution as the one derived in \cite{BayesianGPLVM}. Further calculations in the the $\hat{\mathcal{F}}_v$ term reveal that the optimal setting for $q(\bfu_d)$ is also a Gaussian. More specifically, 
we have:
\begin{align}
\hat{\mathcal{F}}_v={}& \int q(\bfu_d) \log \frac{e^{\la \log N \left( \bfy_d | \bfa_d, \beta^{-1} I_d \right) \ra_{q(X)}}
		p(\bfu_d)}{q(\bfu_d)} d\bfu_d - A \label{boundFAnalytically5}
\end{align}
where $A$ is a collection of remaining terms and $\bfa_d$ is the mean of \eqref{priorF2Suppl}.
\eqref{boundFAnalytically5} is a KL-like quantity and, therefore, $q(\bfu_d)$ is optimally set to be the quantity appearing in the numerator of the above equation. So:
%:
% \begin{equation}
% q(\bff_{*,d}^m | X_*) = \mathcal{N}(\bff_{*,d}^m| \beta K_{N_* M} 
% (K_{MM} + \beta \Psi_2)^{-1} \Psi_1^{T} \bfy_d, K_{N_* N_*} - 
%  K_{N_* M} \left[ K_{M M}^{-1}  - (K_{M M} + \beta \Psi_2)^{-1} \right] 
%  K_{N_* M}^{T}),
% \end{equation}
% exactly as in \cite{BayesianGPLVM}.
\begin{equation}
\label{qu}
q(\bfu_d) = e^{\la \log \mathcal{N} \left( \bfy_d | \bfa_d, \beta^{-1} I_d \right) \ra_{q(X)}}
		p(\bfu_d) ,
\end{equation}
exactly as in \cite{BayesianGPLVM}. This is a Gaussian distribution since $p(\bfu_d ) = \mathcal{N} (\bfu_d | \mathbf{0}, K_{MM} )$.

\par
The complete form of the Jensen's lower bound turns out to be:
\begin{align}
\mathcal{F}_v(q, \boldsymbol \theta) = {}& \sum_{d=1}^{D} 
	\hat{\mathcal{F}}_d(q, \boldsymbol \theta) -  \text{KL}(q \parallel p) \nonumber \\
	= {}& 
	\sum_{d=1}^{D} 
		\log \left( 
		\frac{(\beta)^{\frac{N}{2}} \vert \mathit{K_{MM}} \vert ^\frac{1}{2} }
			 {(2\pi)^{\frac{N}{2}} \vert \beta \Psi_2 + \mathit{K_{MM}}  \vert ^\frac{1}{2} } 	
		 e^{-\frac{1}{2} \mathbf{y}^{T}_{d} W \mathbf{y}_d} 
		 \right) -
		 \frac{\beta \psi_0}{2} + \frac{\beta}{2} 
		 \text{Tr} \left( \mathit{K_{MM}^{-1}} \Psi_2 \right)  \nonumber \\
{}&	- \frac{Q}{2} \log \vert \mathit{K_t} \vert - \frac{1}{2} \sum_{q=1}^{Q}
	  \left[ \text{Tr} \left( \mathit{K_t}^{-1} \mathit{S_q} \right)	  
	  	   + \text{Tr} \left( \mathit{K_t}^{-1} \boldsymbol \mu_q \boldsymbol \mu_q^T \right) \right] 
	 + \frac{1}{2} \sum_{q=1}^Q \log \vert \mathit{S_q} \vert + const  \label{boundFinal}
\end{align}
where the last line corresponds to the KL term. Also:
\begin{equation}
\label{psis}
\Psi_0 = \text{Tr}(\langle \mathit{K_{NN}} \rangle_{q(\mathit{X})}) \;, \;\;
\Psi_1 = \langle \mathit{K_{NM}} \rangle_{q(\mathit{X})} \;, \;\;
\Psi_2 = \langle \mathit{K_{MN}} \mathit{K_{NM}} \rangle_{q(\mathit{X})}
\end{equation}
The $\Psi$ quantities can be computed analytically as in \cite{BayesianGPLVM}.


%-------------------------

\section{Derivatives of the variational bound}
Before giving the expressions for the derivatives of the variational bound \eqref{jensensSuppl},
it should be reminded that the variational parameters $\mu_q$ and $S_q$ (for all $q$s) have been
reparametrized as $S_q = \left( \mathit{K}_t^{-1} + diag(\boldsymbol \lambda_q) \right)^{-1}  \text{ and }   \boldsymbol \mu_q = K_t \bar{\boldsymbol \mu}_q$, where the function $diag(\cdot)$ transforms a vector into a square diagonal matrix and vice versa. Given the above, the set of the parameters to be optimised is 
$( \boldsymbol \theta_f, \boldsymbol \theta_x, \{ \bar{\bfmu}_q, \boldsymbol \lambda_q \}_{q=1}^Q, \tilde{X})$. The gradient w.r.t the inducing points $\tilde{X}$, however, has exactly the same form as for $\boldsymbol \theta_f$ and, therefore, is not presented here. Also notice that from now on we will often use the term ``variational parameters'' to refer to the new quantities $\bar{\bfmu}_q$ and $\boldsymbol \lambda_q$. 

\textbf{Some more notation:} 
\begin{enumerate}
\item $\lambda_q$ is a scalar, an element of the vector $\boldsymbol \lambda_q$ which, in turn, is the main diagonal of the diagonal matrix $\Lambda_q$. 
%\item$\lambda_m \triangleq \boldsymbol \lambda_{q;m}$, i.e. the $m$-th element of the vector $\boldsymbol \lambda_q$ (thus, an instantiation of $\lambda_q$)
\item $S_{ij} \triangleq S_{q;ij}$ the element of $S_q$ found in the $i$-th row and $j$-th column.
\item $\mathbf{s}_q \triangleq \lbrace S_{q;ii} \rbrace_{i=1}^N$, i.e. it is a vector with the diagonal of $S_q$.
%\item $s_i$ is the $i$-th element of $\mathbf{s}_q$.
%\item $diag(\mathbf{s}_q)$ is a matrix full of zeros apart from the main diagonal which contains the vector $\mathbf{s}_q$.
\end{enumerate}

\subsection{Derivatives w.r.t the variational parameters}
\begin{equation}
    \label{derivVarParamSuppl}
\frac{\vartheta \mathcal{F}_v}{\vartheta \bar{\boldsymbol \mu}_q} 
=  K_t \left( \frac{\vartheta \hat{\mathcal{F}}_v}{\vartheta \boldsymbol \mu_q} - \bar{\boldsymbol \mu}_q \right)
\text{ and }
 \frac{\vartheta \mathcal{F}_v}{\vartheta \boldsymbol \lambda_q}
= - ( S_q \circ S_q) \left( \frac{\vv \hat{\mathcal{F}}_v}{\vv \mathbf{s}_q} + \frac{1}{2} \boldsymbol \lambda_q \right).
\end{equation}

where $\circ$ denotes the Hadamard product and:

\begin{align}
 \frac{\hat{\mathcal{F}_v}(q, \boldsymbol \theta)}{\vartheta \mu_q}
{}& = - \frac{\beta D}{2} \frac{\vartheta \Psi_0}{\vartheta \mu_q}
    + \beta \text{Tr} \left(\frac{\vartheta \Psi_1^T}{\vartheta \mu_q} Y Y^T \Psi_1 A^{-1} \right) \nonumber \\
{}& + \frac{\beta}{2} \text{Tr} \left[ \frac{\vartheta \Psi_2}{\vartheta \mu_q}
       \left(
	  D K_{MM}^{-1} - \beta^{-1} D A^{-1} - A^{-1} \Psi_1^T Y Y^T \Psi_1 A^{-1}
       \right) \right] \label{derivFTildeEfficientComputationMu}
\end{align}


\begin{align}
 \frac{\vv \hat{\mathcal{F}_v}(q, \boldsymbol \theta)}{\vartheta S_{q;i,j}}
{}& = - \frac{\beta D}{2} \frac{\vartheta \Psi_0}{\vartheta S_{q;i,j}}
    + \beta \text{Tr} \left(\frac{\vartheta \Psi_1^T}{\vartheta S_{q;i,j}} Y Y^T \Psi_1 A^{-1} \right) \nonumber \\
{}& + \frac{\beta}{2} \text{Tr} \left[ \frac{\vartheta \Psi_2}{\vartheta S_{q;i,j}}
       \left(
	  D K_{MM}^{-1} - \beta^{-1} D A^{-1} - A^{-1} \Psi_1^T Y Y^T \Psi_1 A^{-1}
       \right) \right] \label{derivFTildeEfficientComputationS}
\end{align}


with $A=\beta^{-1}K_{MM}+\Psi_2$.


%-------



\subsection{Derivatives w.r.t $\boldsymbol \theta = (\boldsymbol \theta_f, \boldsymbol \theta_x)$ and $\beta$}
Given that the KL term involves only the temporal prior, its gradient w.r.t the parameters $\boldsymbol \theta_f$ is zero. Therefore:
\begin{equation}
   \label{DerivativeOfFComplete}
      \frac{\vartheta \mathcal{F}_v}{\vartheta \theta_f} = \frac{\vartheta \hat{\mathcal{F}}_v}{\vartheta \theta_f}
\end{equation}

  with:

\begin{align}
\frac{\vartheta \hat{\mathcal{F}}_v}{\vartheta \theta_f} {}& = \text{const} - 
\frac{\beta D}{2} \frac{\vartheta \Psi_0}{\vartheta \theta_f}
 + \beta \text{Tr} \left(\frac{\vartheta \Psi_1^T}{\vartheta \theta_f} Y Y^T \Psi_1 A^{-1} \right) \nonumber \\
{}& + \frac{1}{2} \text{Tr} \left[ \frac{\vartheta K_{MM}}{\vartheta \theta_f}
        \left(
	   D K_{MM}^{-1} - \beta^{-1} D A^{-1} - A^{-1} \Psi_1^T Y Y^T \Psi_1 A^{-1} - \beta D K_{MM}^{-1} \Psi_2 K_{MM}^{-1} 
         \right) \right] \nonumber \\
{}& + \frac{\beta}{2} \text{Tr} \left[ \frac{\vartheta \Psi_2}{\vartheta \theta_f} \;\;\;\;
       \left(
	  D K_{MM}^{-1} - \beta^{-1} D A^{-1} - A^{-1} \Psi_1^T Y Y^T \Psi_1 A^{-1}
       \right) \right] \label{DerivativeOfFtildeComplete}
\end{align}

The expression above is identical for the derivatives w.r.t the inducing points.
For the gradients w.r.t the $\beta$ term, we have a similar expression:



\begin{align}
\frac{\vartheta \hat{\mathcal{F}}_v}{\vartheta \beta} ={}&
  \frac{1}{2} \Big[ 
      D \left( \text{Tr}(K_{MM}^{-1} \Psi_2) + (N-M)\beta^{-1} - \Psi_0 \right) - \text{Tr}(Y Y^\T)
	  + \text{Tr}(A^{-1}\Psi_1^\T Y Y^\T \Psi_1) \nonumber \\
   +{}& \beta^{-2} D \text{Tr} ( K_{MM} A^{-1} ) + \beta^{-1} \text{Tr} \left( K_{MM}^{-1} A^{-1} \Psi_1^\T Y Y^\T \Psi_1 A^{-1} \right) \Big]
\label{derivb2}
\end{align}


In contrast to the above, the term $\hat{\mathcal{F}}_v$ does involve parameters $\boldsymbol \theta_x$, because it involves the variational parameters that are now reparametrized with $K_t$, which in turn depends on $\boldsymbol \theta_x$. 
To demonstrate that, we will forget for a moment the reparametrization of $S_q$ and we will express the bound as $F(\boldsymbol \theta_x, \mu_q (\boldsymbol \theta_x))$ (where $\mu_q (\boldsymbol \theta_x) = K_t \bar{\boldsymbol \mu_q}$) so as to show explicitly the dependency on the variational mean which is now a function of $\boldsymbol \theta_x$. Our calculations must now take into account the term
$
\left( \frac{\vartheta \hat{\mathcal{F}}_v(\boldsymbol \mu_q)}{\vartheta \boldsymbol \mu_q} \right)^\T
       \frac{\vartheta \mu_q (\boldsymbol \theta_x)}{\vartheta \boldsymbol \theta_x}
$
that is what we ``miss'' when we consider $\mu_q(\boldsymbol \theta_x) = \boldsymbol \mu_q$:
\begin{align}
\frac{\vartheta \mathcal{F}_v(\boldsymbol \theta_x, \mu_q(\boldsymbol \theta_x))}{\vartheta \theta_x} = {}&
	\frac{\vartheta \mathcal{F}_v(\boldsymbol \theta_x, \boldsymbol \mu_q)}{\vartheta \theta_x} 
  +  \left( \frac{\vartheta \hat{\mathcal{F}}_v(\boldsymbol \mu_q)}{\vartheta \boldsymbol \mu_q} \right)^\T
            \frac{\vartheta \mu_q(\boldsymbol \theta_x)}{\vartheta \theta_x} \nonumber \\
= {}&
 \cancel{
    \frac{\vartheta \hat{\mathcal{F}}_v(\boldsymbol \mu_q)}{\vartheta \theta_x}
  } +
  \frac{\vv (-\text{KL})(\boldsymbol \theta_x, \boldsymbol \mu_q(\boldsymbol \theta_x))}{\vartheta \theta_x}
+  \left( \frac{\vartheta \hat{\mathcal{F}}_v(\boldsymbol \mu_q)}{\vartheta \boldsymbol \mu_q} \right)^\T
            \frac{\vartheta \mu_q(\boldsymbol \theta_x)}{\vartheta \theta_x}
\label{meanReparamDerivFTheta}
\end{align}

We do the same for $S_q$ and then we can take the resulting equations and replace $\bfmu_q$ and $S_q$ with their equals so as to obtain the final expression which only contains $\bar{\bfmu}_q$ and $\boldsymbol \lambda_q$:

\begin{align}
\frac{\vartheta \mathcal{F}_v(\boldsymbol \theta_x, \mu_q(\boldsymbol \theta_x), S_q(\boldsymbol \theta_x))}{\vartheta \theta_x}
={}& \text{Tr} \bigg[
\Big[ - \frac{1}{2} \left( \hat{B}_q K_t \hat{B}_q + \bar{\bfmu}_q \bar{\bfmu}_q^\T \right) \nonumber \\
+{}& \left( I - \hat{B}_q K_t \right)
 diag \left(  \frac{\vv \hat{\mathcal{F}}_v}{\vv \mathbf{s}_q} \right)
			 \left( I - \hat{B}_q K_t \right)^\T \Big]
			  \frac{\vv K_t}{\vv \theta_x} \bigg] 	\nonumber \\	
+{}&  \left( \frac{\vartheta \hat{\mathcal{F}}_v( \boldsymbol \mu_q)}{\vartheta \boldsymbol \mu_q} \right)^\T
					\frac{\vv K_t}{\vv \theta_x} \bar{\boldsymbol \mu}_q 
\label{CompleteBoundDerivThetatB}
\end{align}
where $\hat{B}_q = \Lambda_q^{\frac{1}{2}} \widetilde{B}_q^{-1} \Lambda_q^{\frac{1}{2}}$.
and $\tilde{B}_q = I + \Lambda_q^{\frac{1}{2}} K_t \Lambda_q^{\frac{1}{2}}$. Note that by using this
$\tilde{B}_q$ matrix (which has eigenvalues bounded below by one) we have an expression which, when implemented, leads to more numerically stable computations, as explained in \cite{rasmussen-williams} page 45-46. 




\section{Predictions}


\subsection{Predictions given only the test time points \label{supplUnobservedData}}
%Firstly, we discuss how the model can predict or generate a set of outputs $Y_*$ given only an input time-vector $\bft_*$. 
To approximate the predictive density, we will need to introduce the underlying latent function values $F_* \in \mathbb{R}^{N_* \times D}$ (the noisy-free version of $Y_*$) and the latent variables $X_* \in \mathbb{R}^{N_* \times Q}$. We  write the predictive density as
\begin{eqnarray}
p(Y_* | Y) & = & \int p(Y_*, F_*, X_*| Y)  \intd  F_* \intd  X_* =  \int p(Y_* | F_*)  p(F_*|X_*, Y) p(X_*|  Y) \intd  F_* \intd  X_* .
\label{eq:predictive1Suppl}
\end{eqnarray}
The term $p(F_* |X_*, Y)$ is approximated according to
\begin{eqnarray}
q(F_*|X_*) & = & \int \prod_{d \in D} p(\bff_{*,d} | \bfu_d, X_*)  q(\bfu_d) \intd  \bfu_d 
	    = \prod_{d \in D} q(\bff_{*,d} | X_*)  ,
\end{eqnarray}
where $q(\bff_{*,d} | X_*)$ is a Gaussian that can be computed analytically , since $q(\bfu_d)$ is also a Gaussian as shown in \eqref{qu}.
%, found after doing some calculations in the $\tilde{\mathcal{F}}_v$ term of \eqref{jensens}.
% $$
% q(\bff_{*,d}^m | X_*) = \mathcal{N}(\bff_{*,d}^m| \beta K_{N_* M} 
% (K_{MM} + \beta \Psi_2)^{-1} \Psi_1^{T} \bfy_d, K_{N_* N_*} - 
%  K_{N_* M} \left[ K_{M M}^{-1}  - (K_{M M} + \beta \Psi_2)^{-1} \right] 
%  K_{N_* M}^{T})
% $$
The term $p(X_*| Y)$ in eq. (\ref{eq:predictive1Suppl}) is approximated by
a Gaussian variational distribution $q(X_*)$,
%
\begin{align}
p(X_* | Y) \approx {}& \int  p(X_* | X) q(X) \intd  X = \la  p(X_* | X) \ra_{q(X)} = q(X_*) = \prod_{q=1}^Q q(\bfx_{*,q}),\label{qxstarSuppl}
\end{align}
%
where $p( X_{*,q} | X)$ can be found from the conditional GP prior
(see \cite{rasmussen-williams}). We can then write
%
\begin{equation}
\label{qxstar2Suppl}
\bfx_{*,q} = \boldsymbol \alpha \bfx_q + \boldsymbol \epsilon,
\end{equation} 
%
where $\boldsymbol \alpha = K_{*N}K_t^{-1}$ and 
$\boldsymbol \epsilon \sim \mathcal{N} \left( \bfz, K_{**} - K_{*N K_t^{-1} K_{N*}}\right)$. Also, $K_t = k_x(\bft, \bft)$, $K_{*N} = k_x(\bft_*, \bft)$ and $K_{**} = k_x(\bft_* \bft_*)$. 
Given the above, we know a priori that \eqref{qxstarSuppl} is a Gaussian and by taking expectations over $q(X)$ in the r.h.s. of \eqref{qxstar2Suppl} we find the mean and covariance of $q(X_*)$. Substituting for the equivalent forms of $\bfmu_q$ and $S_q$ from section \ref{optimisation} we obtain the final solution
%
\begin{align}
 \mu_{x_{*,q}} = {}& \bfk_{*N} \bar{\mu}_q \\
  \text{var}(x_{*,q}) = {}& k_{**} - \bfk_{*N} (K_t + \Lambda_q^{-1})^{-1} \bfk_{N*}.
\end{align}
%
\eqref{eq:predictive1Suppl} can then be written as:
%`
\begin{align} 
p(Y_*| Y) {}& =  \int p(Y_*| F_*)  q(F_*|X_*) q(X_*) \intd  F_* \intd  X_* = \int p(Y_* | F_*) \la q(F_* | X_*) \ra_{q(X_*)} \intd  F_* \label{eq:predictive2Suppl}
\end{align}
%
Although the expectation appearing in the above integral is not a Gaussian, its moments can be found analytically \cite{rasmussen-williams, Girard03gaussianprocess},
%
\begin{align}
 \mathbb{E}(F_*) ={}&  B^\T \Psi_1^* \label{meanFstarSuppl} \\
 \text{Cov}(F_*) ={}& B^\T \left( \Psi_2^* - \Psi_1^* (\Psi_1^*)\T \right) B + \Psi_0^* I - \text{Tr} \left[ \left( K_{MM}^{-1} - \left( K_{MM} + \beta \Psi_2 \right)^{-1} \right) \Psi_2^* \right] I,
\end{align}
%
where $B = \beta \left( K_{MM} + \beta \Psi_2 \right)^{-1} \Psi_1^\T
Y$, $\Psi_0^* = \la k_f(X_*, X_*) \ra$, $\Psi_1^* = \la K_{M*} \ra$
and $\Psi_2^* = \la K_{M*} K_{*M} \ra$. All expectations are taken
w.r.t. $q(X_*)$ and can be calculated analytically, while $K_{M*}$
denotes the cross-covariance matrix between the training inducing
inputs $\tilde{X}$ and $X_*$. Finally, since $Y_*$ is just a noisy version of
$F_*$, the mean and covariance of \eqref{eq:predictive2Suppl} is just
computed as: $\mathbb{E}(Y_*) = \mathbb{E}(F_*)$ and $\text{Cov}(Y_*)
= \text{Cov}(F_*) + \beta^{-1} I_{N_*}$.


\subsection{Predictions given the test time points and partially observed outputs}

The expression for the predictive density $p(Y_*^m | Y_*^p, Y)$ follows exactly as in section \ref{supplUnobservedData} but we need to compute probabilities for $Y_*^m$ instead of $Y_*$ and $Y$ is replaced with $(Y, Y_*^p)$ in all conditioning sets. Similarly, $F$ is replaced with $F^m$. Now $q(X_*)$ cannot be found analytically as in section \ref{supplUnobservedData}; instead, it is optimised so that $Y_*^p$ are taken into account. 
This is done by maximising the variational lower bound on the marginal likelihood:
\begin{align}
p(Y_*^p, Y) ={}&  \int p(Y_*^p, Y|X_*, X) p(X_*, X) \intd  X_* \intd  X \nonumber \\
			={}&  \int p(Y^m | X) p(Y_*^p, Y^p|X_*, X) p(X_*, X) \intd  X_* \intd  X,  \nonumber
\end{align}  
Notice that here, unlike the main paper, we work with the likelihood after marginalising $F$, for simplicity.
Assuming a variational distribution 
$q(X_*, X)$ and using Jensen's inequality we obtain the 
lower bound 
\begin{eqnarray}
& & \int q(X_*, X) \log \frac{ p(Y^m | X) 
p(Y_*^p, Y^p|X_*, X) p(X_*,X)}{ q(X_*, X)} \intd  X_* \intd  X \nonumber \\ 
& = & \int q(X) \log p(Y^m | X) \intd  X 
+  \int q(X_*,X) \log p(Y_*^p, Y^p|X_*, X) \intd  X_* \intd  X  \nonumber \\
& - & \text{KL}[q(X_*,X) || p(X_*, X)] \label{partialPredLowerBoundSuppl}
\end{eqnarray}  
%
This quantity can now be maximized in the same manner as for the bound
of the training phase. Unfortunately, this means that the variational
parameters that are already optimised from the training procedure
cannot be used here because $X$ and $X_*$ are coupled in $q(X_*,X)$. A
much faster but less accurate method would be to decouple the test
from the training latent variables by imposing the factorisation
$q(X_*, X) = q(X) q(X_*)$. Then, equation
\eqref{partialPredLowerBoundSuppl} would break into terms containing $X$,
$X_*$ or both. The ones containing only $X$ could then be treated as
constants.


\section{Additional results from the experiments}
\begin{figure}[ht]
\begin{center}
\subfigure[]{
	\includegraphics[width=0.4\textwidth]{../diagrams/supplMocapScalesRbf}
	\label{fig:suppMocap1}
}
\subfigure[]{
	\includegraphics[width=0.4\textwidth]{../diagrams/supplMocapScalesMatern}
	\label{fig:suppMocap2}
}
\end{center}
\caption{\small{
The values of the scales of the ARD kernel after training on the motion capture dataset using the RBF (fig: \subref{fig:suppMocap1}) and the Mat\'ern (fig: \subref{fig:suppMocap2}) covariance function to model the dynamics for VGPDS. The scales that have zero value ``switch off'' the corresponding dimension of the latent space. The latent space is, therefore, 3-D for \subref{fig:suppMocap1} and 4-D for \subref{fig:suppMocap2}. Note that the scales were initialized with very similar values.
}
}
\label{fig:supplMocap1}
\end{figure}


\begin{figure}[ht]
\begin{center}
\subfigure[]{
	%\includegraphics[width=0.48\textwidth]{../diagrams/supplMocapBody23GpdsRbf}
	\includegraphics[width=0.48\textwidth]{../diagrams/supplMocapBody28GpdsMatern}
	\label{fig:suppMocap3}
}
\subfigure[]{
	\includegraphics[width=0.48\textwidth]{../diagrams/supplMocapLeg5GpdsMatern}
	\label{fig:suppMocap4}
}
\end{center}
\caption{\small{
The prediction for two of the test angles for the body (fig: \ref{fig:suppMocap3}) and for the legs part (fig: \ref{fig:suppMocap3}). Continuous line is the original test data, dotted line is nearest neighbour in scaled space, dashed line is VGPDS (using the RBF covariance function for the body reconstruction and the Mat\'ern for the legs).
}
}
\label{fig:supplMocap2}
\end{figure}




\begin{figure}[ht]
\begin{center}
\subfigure[]{
	\includegraphics[width=0.23\textwidth]{../diagrams/supplDogPredYts5}
	\label{fig:suppDog1}
}
\subfigure[]{
	\includegraphics[width=0.23\textwidth]{../diagrams/supplDogPredGpds5}
	\label{fig:suppDog2}
}
\subfigure[]{
	\includegraphics[width=0.23\textwidth]{../diagrams/supplDogPredYts6}
	\label{fig:suppDog3}
}
\subfigure[]{
	\includegraphics[width=0.23\textwidth]{../diagrams/supplDogPredGpds6}
	\label{fig:suppDog4}
}
\end{center}
\caption{\small{
 Some more examples for the reconstruction achieved for the `dog' dataset. $40\%$ of the test image's pixels (figures \subref{fig:suppDog1} and \subref{fig:suppDog3}) were presented  to the model, which was able to successfully reconstruct them, as can be seen in \subref{fig:suppDog2} and \subref{fig:suppDog4}.
}
}
\label{fig:supplDog}
\end{figure}



\end{document}
